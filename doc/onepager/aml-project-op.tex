\documentclass[11pt, a4paper]{article}
\usepackage[utf8]{inputenc}
\usepackage[english]{babel}
\usepackage{geometry}
\usepackage{graphicx}
\usepackage{helvet}
\usepackage{enumitem}

% Page margins to fit on one page
\geometry{
 a4paper,
 left=2cm,
 right=2cm,
 top=2cm,
 bottom=2cm
}

% Font (modern, sans-serif)
\renewcommand{\familydefault}{\sfdefault}

% No paragraph indentation
\setlength{\parindent}{0pt}
% Space between paragraphs
\setlength{\parskip}{1.0ex}

%-------------------------------------------------
% TITLE SECTION
%-------------------------------------------------

\title{{\textbf{VegDetect: Real-Time Detection of Vegan Product Labels Using Compact AI Models on Edge Devices}}}
\author{\small{Lorenz Hund - Student ID: 3138947} \\ \small{Advanced Machine Learning Project WS25/26}}
\date{} % No date under title

%-------------------------------------------------
% DOCUMENT START
%-------------------------------------------------
\begin{document}

\maketitle
\pagestyle{empty} % <-- NEU: Entfernt die Seitenzahl

%-------------------------------------------------
\section*{1. Motivation}
%-------------------------------------------------

Identifying vegan products in retail stores is often a time-consuming process for consumers, requiring them to search for small labels or study ingredient lists. 
Furthermore, according to European Parliament concerns, product names like "vegan sausage" apparently cause massive confusion and pose a serious threat to consumers. 
To counter this (almost) life-threatening deception and simplify the shopping process, this project will develop an AI-based real-time solution.

The goal is a compact system that runs directly on an edge device (e.g., laptop or smartphone) to detect the official "V-Label" (vegan) using the camera.

%-------------------------------------------------
\section*{2. Dataset \& Methodology}
%-------------------------------------------------

The project will utilize the publicly available \textbf{NutriGreen Dataset}\footnote{Drole, J., Pravst, I., Eftimov, T., \& Seljak, B. K. (2023). NutriGreen Image Dataset: A Collection of Annotated Nutrition, Organic, and Vegan Food Products (Version 1.0) [Data set]. Zenodo. https://doi.org/10.5281/zenodo.8374047}. This dataset is ideal as it already provides thousands of images of food packaging, including annotations for the V-Label in YOLO format, eliminating the need for manual data collection or labeling.

The core of the project is the evaluation of "one-stage" object detection architectures. These models are known for their high inference speed and are therefore ideal for edge applications.

\begin{itemize}[leftmargin=*]
 \item \textbf{Model Candidates:} Various lightweight architectures will be compared (e.g., \textbf{YOLOv8-Nano}, \textbf{EfficientDet-Lite}, or \textbf{MobileNet-SSD}).
 \item \textbf{Training:} Training will be conducted via transfer learning, where pre-trained models are fine-tuned for the specific task (detecting the V-Label class).
 \item \textbf{Deployment:} The trained models will be exported to an efficient inference format to maximize performance on a CPU.
\end{itemize}

%-------------------------------------------------
\section*{3. Project Goals}
%-------------------------------------------------

This project has two primary objectives:

\begin{enumerate}[leftmargin=*]
 \item \textbf{Functional Prototype:} Development of an application that demonstrates real-time V-Label detection on products using a standard webcam (live feed).
 \item \textbf{Benchmark Analysis:} Performing a systematic evaluation of the trade-off between inference speed (Frames Per Second - FPS) and detection accuracy (mean Average Precision - mAP) of the evaluated models on a reference laptop (CPU inference).
\end{enumerate}
\end{document}